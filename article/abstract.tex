\begin{abstract}
% Normalizing flows use a sequence of invertible transformations $f$ to transform a simple distribution into a more complex target distribution.
% The key defining property of flow-based models is that the transformations must be invertible and both $f$ and $f^{-1}$ must be differentiable.
Normalizing flows based on coupling layers require a bijective one-dimensional function and the derivative of the function with respect the input variable.
Related flows based on coupling layers such as NICE and RealNVP have an analytic one-pass inverse, but are often less flexible than their autoregressive counterparts. 
Based on these limitations, this article proposes to implement the coupling function using the integration of continuous piecewise-affine (CPA) velocity functions as a building block. 
The module acts as a drop-in replacement for the affine or additive transformations commonly found in coupling and autoregressive transforms. 
When combined with alternating invertible linear transformations, the resulting class of normalizing flows is referred to as closed-form diffeomorphic spline flows (DIFW-NF), which may feature coupling layers, DIFW-NF (C), or autoregressive layers, DIFW-NF (AR).
Experiments demonstrate that this module significantly enhances the flexibility of both classes of flows, obtaining competitive results in a variety of high-dimensional datasets.
Unlike the additive and affine transformations, which have limited flexibility, the proposed differentiable monotonic function with sufficiently many intervals can approximate any differentiable monotonic function, yet has a closed-form, tractable Jacobian determinant, and can be inverted analytically. Our parameterization is fully-differentiable, which allows for training by gradient methods.
\end{abstract}